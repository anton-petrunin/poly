\documentclass{article}
\usepackage{poly}

\begin{document}

\title{Faceting Riemannian manifolds}
\author{Anton Petrunin}
\date{}
\maketitle


\begin{abstract} 
We give a description of the curvature tensors of Riemannian manifolds which admit Lipschitz approximation by polyhedral metrics with curvature bounded below or above.
We show that the described curvature condition is also sufficient for existence of local approximations.

We conjecture that this condition is also sufficient for the global approximations and prove it in some special cases.
\end{abstract}


%\addtocounter{section}{-1}
\section*{Introduction}

\parbf{Positive cosectional curvature.}
Let $\EE^m$ denotes  $m$-dimensional Euclidean space.
Denote by $\A^4(\EE^m)$ the space of all curvature tensors of Riemannian manifolds on the tangent space $\EE^m$.
The $\O(m)$-rotations of $\EE^m$ induce isometric rotations of space of tensors  $\A^4(\EE^m)$.
The subset of $\A^4(\EE^m)$ will be called \emph{$\O(m)$-invariant} if it is invariant with respect to these rotations.

Let $Q_{\SS^2\times \RR^{m-2}}$ be the curvature tensor of $\SS^2\times \RR^{m-2}$.
Denote by $\mathcal{S}^*$ the minimal convex $\O(n)$-invariant cone in $\A^4(\EE^m)$ that contains $Q_{\SS^2\times \RR^{m-2}}$.

Let $M$ be a Riemannian manifold and $p\in M$;
denote by $\Rm_p$ the curvature tensor of $M$ at $p$.
We say that \emph{cosectional curvature} of $M$ at $p$ is at least (at most) $\kappa$
if $\Rm_p-\kappa\cdot Q_{\SS^m}\in\mathcal{S}^*$, 
or correspondingly $-\Rm_p-\kappa \cdot Q_{\SS^m}\in\mathcal{S}^*$,
where $Q_{\SS^m}$ denotes the curvature tensor of unit $m$ sphere $\SS^m$.
Briefly, these conditions can be written as $\cosec(\Rm_p)\ge \kappa$ and $\cosec_p\ge \kappa$ or correspondingly $\cosec(\Rm_p)\le \kappa$ and $\cosec_p\le \kappa$.
The reason for the name \emph{cosectional} will become clear below, see 1.E.

We will write $\cosec(M)\ge \kappa$ (or $\cosec(M)\le \kappa$) if $\cosec_p\ge \kappa$ (or, correspondingly, $\cosec_p\le \kappa$) for all $p\in M$.


The following theorem gives an ``if and only if'' condition for the existance of local approximations of Riemannian manifold by polyhedral spaces with upper or lower curvature bound.

Recall that a metric space is called \emph{$\kappa$-polyhedral spaces} if it admits a triangulation such that each simplex is isometric to a simplex in the model space of curvature $\kappa$.
The $0$-polyhedral spaces are called \emph{Euclidean polyheral spaces},
the $1$-polyhedral spaces are called \emph{spherical polyheral spaces}
and
the $(-1)$-polyhedral spaces are called \emph{hyperbolic polyheral spaces}.

\begin{thm}{Local Theorem}
 Let $P_n$ be a sequence of $m$-dimensional $\kappa$-polyhedral spaces
with curvature $\ge \kappa$ (or $\le\kappa$) in the sense of Alexandrov.
Assume $P_n$ Lipschitz converge to a Riemannian manifold $M$, then $\cosec(M)\ge \kappa$ (correspondingly $\cosec(M)\le \kappa$).

Moreover if $M$ is a Riemannian manifold with $\cosec(M)\ge\kappa+\eps$ for some $\eps>0$
then each point of $M$ has a neighborhood which is a Lipschitz limit of a sequence of
polyhedral spaces with curvature $\ge \kappa$.
\end{thm}


\begin{thm}{Global Theorem} 
Let $M$ be Riemannian $m$-manifold with $\cosec(M)\z\ge \kappa+\eps$ for some $\eps>0$.
Assume that $M$ (or its finite cover) has
stably trivial tangent bundle.
Then $M$ admits a Lipschitz approximation
by a sequence of $m$-dimensional $\kappa$-polyhedral metrics with curvature
$\ge \kappa$.
\end{thm}

We expect that the condition on the tangent bundle can be
removed from this formulation.
The simplest interesting example which is not covered by the theorem is $\CP^2$ ---
the complex projective plane with canonical metric.
While
$\cosec\CP^2\ge 0$,
I can not construct an approximation of $\CP^2$ by polyhedral metric with curvature $\ge -\epsilon$.
Also, I can not show that there is no such approximations with curvature $\ge 0$.
Number of examples of polyhedral metrics on $\CP^2$ with curvature $\ge 0$ were constructed in [Pan].

The Global Theorem above can be reduced to the cases $\kappa= \{-1,0,1\}$,
in the first and last case using rescaling one can get an approximation
of $(M,g)$ with polyhedral metrics with curvature $\ge \kappa$.
In case $\kappa=1$, the condition $\cosec M\ge 1$ implies, in particular, that curvature
operator of $M$ is strictly positive (see 1.E).
In particular, from Micallef--Moore theorem [MM] it follows that universal cover $\widetilde M$ must be homeomorphic (and by B\"ohm--Wilking theorem [BW] diffeomorphic) to
a sphere.
In particular, $\widetilde M$ has stably trivial tangent bundle and therefore
we get the following:


\begin{thm}{Corollary} 
An $m$-manifold $M$ admits a Lipschitz approximation by
spherical polyhedral metrics with curvature $\ge 1$ 
if and only if $\cosec(M)\ge 1$.
\end{thm}

\parbf{About the proofs.}
The necessity of the curvature bound follows from the fact that all curvature
of a polyhedral metric lives on hyper-edges;
that is, the simplexes of codimension $2$ of the triangulation of the polyhedral space.
Around every hyper-edge the metric looks like $C\times \RR^{m-2}$,
where $C$ is a two-dimensional cone.
Thus the ``curvature'' at the vertex of $C$
looks pretty much like curvature of $\SS^2$ with zero radius,
 and this allows us to view the ``curvature'' at the edge as curvature of
$\SS^2\times \RR^{m-2}$ (which is $(x\wedge y)^2$) multiplied by a
Hausdorff measure of edge.
 When a space is approximated by polyhedral metrics the curvature
tensors of different edges could mix with each other;
that is, the limit
manifold must have curvature tensor
which is convex combination of the curvatures of above form,
in other words, it will satisfy condition
 $\cosec M\ge 0$.

It gives only an idea, the real proof contains much of technical work; it was recently done by Nina Lebedeva [Leb].
This part of the proof is not included in this paper but I include the unpublished result of Perelman on which the this part of the proof is based in the Appendix B.

To prove the sufficient condition, we construct an isometric embedding $(M^m,g)\hookrightarrow\RR^q$ such that its image is an intersection of $q-m$ open convex hyper-surfaces with obtuse angles between their outer normals at any point of $M$.
This condition on the angles alone implies $\cosec M>0$.

Next we consider an approximation of convex hyper-surfaces
by convex polyhedral hyper-surfaces with the same condition on the angles between the outer normals.
The needed polyhedral approximation is the intersection of these
polyhedral hyper-surfaces. 
That proves the local theorem.

Now let us describe the idea for Global Theorem.
For simplicity assume that $M$ is simply connected.
Using that $\T\,M$ is stably trivial we realize $M$ as an intersection of open convex hyper-surfaces with
the same conditions on angles and then do the same approximation as above.
The proof of this last representation is technical. 
If such representation exists, we have that $\N\, M$, the normal bundle of
$M$, is trivial; in particular the tangent bundle $\T\, M$ should
be stably trivial.
The latter explains how the condition on the tangent bundle in the Global Theorem comes into the game.

\medskip

The main part of this work was done during my stay at the IHES in 1999--2000. I would like to thank this institute for support and hospitality.
I want to thank 
G.~Perelman for sharing ideas and making me interested in this problem long before this publication; 
V.~Voevodsky for bringing paper
of D.~Hilbert to my attention. 
I want to express my very special thanks to J.~Eschenburg and S.~Kozlov who constructed for me
examples of curvature tensors and pulled me out from a
dead end in this research and to R. Matveyev and D. Panov for helpful and interesting conversations and letters.


\section{Notation, Definitions and Preliminaries}



\parbf{1.A. Polyhedral spaces with curvature bounded below or above.}


A connected simplicial complex $P$ is called \emph{pseudomanifold}
if the link of each simplex in $P$ is connected or $\SS^0=\{-1,1\}$.

Let us denote by $\MM^q[\kappa]$ the $q$-dimesional simply connected space of constant curvature $\kappa$.
A pseudomanifold $P$ equiped with a metric 
such that each
simplex is isometric
to a simplex in $\MM^m[\kappa]$
is called
\emph{$\kappa$-polyhedral space}. 

A $\kappa$-polyhedral space has curvature bounded below  if  the sum of angles
around any hyper-edge (that is, simplex of codimension $=2$) is $\le 2\pi$.
In this case the polyhedral space has curvature  $\ge \kappa$ in the sense of Alexandrov.

In all what follows we will assume $\kappa=0$,
but if it is not specially
mentioned everything below is true for any $\kappa$;
one ony has exchange $\RR^q$ to $\MM^q[\kappa]$.

%???NEG curv???


\parbf{1.B. Convex submanifolds of higher codimension.}


\begin{thm}{Definition}
 A submanifold $M \i \RR^q$ is called
\emph{locally convex} if each point of $M$ has a neighborhood $U$, for which
there is a collection of
 (strictly) convex, possibly open, hyper-surfaces $F_i$,
such that $U=\bigcap_i F_i$ and moreover at each point of $U$ the angle between
outward normals to any pair of $F_i$ is obtuse.
\end{thm}


If $M$ is $C^2$-smooth then the above property is equivalent to the following condition:
 at each point $x\in M$ there is an orthonormal
basis $\{e_i\}\i \N_x M$, where $\N M$
is normal bundle of $M\i \RR^q$, such that the representation of
the second fundamental form
$s\:\S^2(\T_xM)\to \N_x M$ in this basis $s=\sum_i e_i s_i$
has all quadratic forms $s_i\in \S^2(\T)$ positively defined.

This property can be also interpreted as the following: the submanifold is
locally convex if it can be viewed locally as a convex hyper-surface
in a convex hyper-surface in ... in $\RR^q$.

It is easy to see that locally convex submanifold
has positive curvature
(for the induced intrinsic metric in the sense of Alexandrov).
If the submanifold is smooth,
one can say  more about its curvature tensor,
but to make it precise we must discuss a little the
curvature tensor of Riemannian manifold and curvature of submanifolds
in $\RR^q$.


\parbf{1.C. Curvature tensors of submanifolds.}
Here we introduce an extrinsic curvature for submanifolds.
The source for this subsection is [Grom]$_{\text{PDR}}$ 3.1.5.


Let $\T$ be a vector space with a scalar product, $\T^n$ will denote
its tensor power of degree $n$, $\S^n(\T)$ and $\Lambda^n(\T)$ will
denote correspondingly subspace of symmetric and antisymmetric
elements of $\T^n$. The scalar product on $\T$ canonically induces a
scalar product on $\T^n$ and all its subspaces.

The following subspace of $\T^4$
$$\A^4(\T)=\Lambda^4(\T)^\bot\cap \S^2(\Lambda^2(\T))$$
formed by all possible curvature tensors
on the tangent space $\T$.
In an equivalent way this subspace $\A^4\i \S^2(\Lambda^2(\T))$
can be described  as the space of all tensors in $\S^2(\Lambda^2(\T))$
satisfying the first Bianchi identity
$$\Rm(X,Y,Z,W)+\Rm(Y,Z,X,W)+\Rm(Z,X,Y,W)=0.$$
In particular, it does not depend on the choice of scalar product on $\T$.

\



Let $M\i \RR^q$ be a smooth submanifold and $s_x\:\S^2(\T_xM)\to \N_x M$
is its second
fundamental form at $x\in M$,
here $\T\,M$ and $\N\, M$ are respectively tangent and normal bundle over $M$.
Consider the \emph{$\Phi$-curvature tensor}
$$\Phi (X,Y,Z,W)= \langle s(X,Y),s(Z,W)\rangle,$$
here $\Phi$ is a section of $\S^2(\S^2(\T M)$.

Tensor $\Phi$ admits the representation
$$\Phi(X,Y,Z,W)=E(X,Y,Z,W)+{\frac 1 3}\biggl(\Rm(X,Z,Y,W)+\Rm(X,W,Y,Z)\biggr)
\eqno(*)$$ 
where $E$ is total symmetrization of $\Phi$; that is,
$$E(X,Y,Z,W)={\frac 1 3}
\biggr(\Phi(X,Y,Z,W)+\Phi(Y,Z,X,W)+\Phi(Z,X,Y,W)\biggl)\in \S^4(\T)$$
 and
$$\Rm(X,Y,Z,W)=\Phi(X,Z,Y,W)-\Phi(X,W,Y,Z)\in \A^4(\T)$$
is the Riemannian curvature tensor of $M$.

Tensor
$E$ represent \emph{ extrinsic curvature},
$E\in \S^4(\T)\i \S^2(\S^2(\T))$ and it
behaves as an entropy of the embedding,
the more the embedding is wrinkled the bigger $E$ gets.
Note that
$f(X)=E(X,X,X,X)=|s(X,X)|^2$
is homogeneous polynomial of degree $4$ on $\T$ 
and it describes $E$ completely.

There are two reasons to use tensors $\Phi$ and $E$: 
\begin{itemize}
\item The tensor tensors $\Phi$ depends only on elements of $\T$, 
in particular it does not depend even on dimension of ambient space.
This way we can study isometric embeddings without direct referring to the ambient space.
\item Direct
construction shows that $\Phi$ describes the second fundamental
form $s\:\S^2(\T)\to \N$
up to an isometric rotation of $\N$;
that is, two second
fundamental forms $s_1,s_2\:\S^2(\T)\to \N$ 
give the same $\Phi\in\S^2(\S^2(\T))$ if and only if there is an isometric rotation
$j\:\N\to \N$, such that $j\circ s_1=s_2$. 
In particular, since
$\Phi$ is a sum of Riemannian curvature tensors and $E$, we have
that if $(M,g)$ is a Riemannian manifold and $(M,g)\to \RR^q$ is
an isometric embedding then $E$-tensor together with $g$ describes
the second fundamental form at each point up to an isometric
rotation.
\end{itemize}




\parbf{1.D. Positiveness of elements of $\S^2(\S^2(\T))$
and convexity of submanifold.}
Most of this subsection is extracted from [Grom]$_{\text{PDR}}$ 2.4.9B(4).

Given a open convex cone $\mathcal{C}$
in an Euclidean space $\EE^n$, set
$$\mathcal{C}^*=\set{r\in \EE^n}{\langle r,r'\rangle >0 \text{ for all }r'\in \mathcal{C}}.$$

\begin{thm}{Definition} A tensor $\Phi\in \S^2(\S^2(\T))$
is positive ($\Phi> 0$), if there is a
representation $\Phi=\sum_i s_i^2$, where $s_i$ are positively defined
quadratic forms on $\T$.
If $i\:M\to \RR^q$ is a smooth embedding we will write
$\Phi(i)>0$ if $\Phi$-tensor of $i(M)\i \RR^q$ is positive
at $i(x)$ for all $x\in M$.
\end{thm}


The cone of positive tensors in $\S^2(\S^2(\T))$ form a convex
$\GL(\T)$-invariant cone of tensors;
that is, this cone is invariant with respect to the action of $\GL(\T)$ induced on $\S^2(\S^2(\T))$.
If $\text{dim }\T\ge 2$ then
there are other $\GL(\T)$ invariant cones in $\S^2(\S^2(\T))$, one of
these cones will be of particular interest for me: namely the cone
of all elements $\Phi=\sum_i s_i^2$ for arbitrary elements $s_i\in
\S^2(\T)$. This cone describes all elements of $\S^2(\S^2(\T))$ which
can appear as $\Phi$-curvature of a submanifold.


Note that existence of representation of the second fundamental form
$s\z=\sum_i s_ie_i$ with
positively defined $s_i\in \S^2(\T)$
implies, in particular, that $\Phi\z=\sum_i s_i^2$;
that is, $\Phi> 0$.
Since $\Phi$-tensor describes the second fundamental form completely the
last property is equivalent to the fact that
$M\i \RR^q$ is ``stably'' locally convex.
Namely, if  $\Phi>0$ on $M$ then
for some $k$ we have
that correspondent submanifold $M\i \RR^q=\RR^q\times 0\i \RR^q\times \RR^k$
is locally convex, as a submanifold in $\RR^q\times \RR^k$.
(Given $k\in \NN$ there are examples of submanifold $M\i \RR^q$
which is not convex as a submanifold $M\i \RR^q\times 0\i \RR^q\times \RR^k$
but is convex as a submanifold $M\i \RR^q\times 0\i \RR^q\times \RR^{k+1}$.)


\parbf{1.E. Positiveness of curvature tensor and symmetric $4$-tensors.}
The space $\S^2(\S^2(\T))$ splits into two subspaces,
the first is $\S^4(\T)\i \S^2(\S^2(\T))$ and the second is
$\A^4_+(\T)=\S^4(\T)^\bot\cap \S^2(\S^2(\T))$
which is canonically isomorphic to space of algebraic curvature tensors
$\A^4(\T)=\Lambda^4(\T)^\bot\cap \S^2(\Lambda^2(\T)$.



$\A^4(\T)$. 
Consider cone $\mathcal{S}^*$ which consists of all tensors
$$\Rm(X,Y,Z,T)=\sum_i s_i(X,Z)s_i(Y,T)-s_i(X,T)s_i(Y,Z),$$ where
$s_i$  are positive elements of $\S^2(T)$. 
We say that such
curvature tensors have positive cosectional curvature;
this can be written as $\cosec (\Rm)>0$. 
For a Riemannian manifold $M$ we will write $\cosec (\Rm_p)>0$ or $\cosec_p>0$ if
curvature tensor at $p\in M$ has positive cosectional curvature and
$\cosec M>0$ if the cosectional curvature of $M$ is positive at
all $p\in M$.

The curvature tensors with positive cosectional curvature are exactly
those which can be curvature tensors of submanifolds with positive
$\Phi$-curvature (or equivalently strictly convex submanifolds see 1.D);
that is,
\begin{align*}
\cosec(&\Rm)>0
\\
&\Updownarrow
\\
\Rm(X,Y,Z,W)=\Phi(X,Z,Y,W)&-\Phi(X,W,Y,Z) \text{ for
some } \Phi>0.
\end{align*}
As you will see, any closed Riemannian
manifold $M$ with $\cosec M>0$ admits a smooth isometric embedding
$i\:M\to \RR^q$ with $\Phi(i)>0$ (in fact, $q$ can be taken
$=(n+2)(n+5)/2$, see [Grom]${}_{\text{PDR}}$ 3.1.5(A) and
3.1.2(C)]). 
Cone $\bar{\mathcal{S}}^*$, the closure of $\mathcal{S}^*$ can also be
described as minimal convex $O(\T)$-invariant cone which contains
curvature tensor of product metric space $\SS^2\times \RR^{n-2}$.

The dual cone $\mathcal{S}$ (see 1.D) consists of curvature tensors with
positive sectional curvature.
For a point $p$ in a Riemannian manifold,
we will write $\sec_p>0$ or $\sec(\Rm_p)>0$ meaning
that $\Rm_p\in \mathcal{S}$.


The cones $\mathcal{S}^*$ and $\mathcal{S}$ are the smallest and biggest
$\GL(\T)$-invariant cones in $\A^4(\T)$. Other $\GL(\T)$-invariant cones
will lie between $\mathcal{S}^*$ and $\mathcal{S}$.
In particular the cone $\mathcal{Q}$ of
all curvature tensors with positive curvature operator;
that is,
$$\mathcal{Q}=\set{R\in \A^4(\T)\i \S^2(\Lambda^2(\T))}{R=\sum_i\phi_i^2 \text{
for } \phi_i\in \Lambda^2(\T)},$$ is one of them.
If the dimension is big enuf, then $\mathcal{Q}\not=\mathcal{S}^*$.
Namely if dimension $=2$ or $3$
then $\mathcal{Q}=\mathcal{S}^*$ and therefore $\mathcal{S}=\mathcal{Q}=\mathcal{S}^*$.
In dimension $4$, we have $\mathcal{Q}=\mathcal{S}^*$ and
$\mathcal{Q}^*=\mathcal{S}$.
It follows from the \emph{Thorpe's characterization} of curvature tensors with positive
sectional curvature,  namely, if $M$ is positively curved
$4$-manifold then there is a function $f$ on $M$ such that
$\Rm_x+f(x)\omega\in \S^2(\Lambda^2(\T_x))$ is a section of positive
quadratic forms on $\Lambda^2(\T)$, here $\omega$ denotes the
volume form, a section of $\Lambda^4(\T)\i \S^2(\Lambda^2(\T))$, see
[Zol] for details.

In dimension $5$ and higher, all the inclusions 
\[\mathcal{S}\subset \mathcal{Q}^*\subset\mathcal{Q}\subset\mathcal{S}^*\]
are strict.
Indeed: evidently, the inclusion $\mathcal{Q}^*\subset\mathcal{Q}$ is strict.
The inclusion $\mathcal{S}\subset \mathcal{Q}^*$ is strict if and only if so is
$\mathcal{Q}\subset\mathcal{S}^*$.
The later is shown by example, see
[Zol].

(In [Grom]$_{SGMC}$, Gromov states opposite. He writes \emph{ ``The closer of this cone (given by $Q\ge 0$) [this is, $\bar{\mathcal{Q}}$ in our notations] can be defined as the minimal
closed convex $O(n)$-invariant cone which contains the curvature
of the product metric on $\SS^2\times \RR^{n-2}$ [which is, $\bar
{\mathcal{S}}^*$ in our notations].''}...)

\

$\S^4(\T)$. Consider cone $C_+$ which we will call cone of \emph{
positive forms}. It consists of all forms $E\in \S^4(\T)$ such that
$E=\sum s_i^{\circ 2}$, where $s_i^{\circ 2}$ is symmetric square
of positively definite quadratic form $s_i$. We will write $E>0$
if $E\in C_+\i \S^4(\T)$. Again, tensor $E$ is positive if it is a
symmetric part of positive $\Phi$-tensor in
$\S^2(\S^2(\T))$.\footnote"$^\bigstar$" {By the way $C_+$ is also the
smallest $\GL(\T)$-invariant cone in $\S^4(\T)$. The biggest such cone
$C^*_+$ consists of all symmetric $4$-form $E$ such that
$$E(X,X,X,X)>0$$ for any non zero $X\in \T$. 
Gromov in
[Grom]${}_{\text{PDR}}$ 3.1.4 states that a symmetric form in
$E\in \S^{2k}(\T)$ is positive if and only if correspondent
quadratic form $E(\S^2(\T^k))\to \RR$ is positively defined.
In our notations, it is equivalent to $C_+=Q_+$, and
this is equivalent to $C_+^*=Q_+^*$. 
The cone $C_+^*$ is nothing
but set of all positively defined forms in $\S^{2k}(\T)$,
equivalently it is set of positively defined homogeneous degree
$2k$ polynomials on $\T$. Analogously the cone $Q_+^*$ is the set
of homogeneous degree $2k$ polynomials on $\T$ which can be
expressed as sum of squares. Therefore this statement is
equivalent to the fact that \emph{ each positively defined
polynomial is a sum of squares of polynomials}, and this was shown
to be wrong in general, namely Hilbert [Hil] showed that this
statement is true ONLY in the following three cases: (i)
$\text{dim}\T\le 2$ and any $k$, (ii) $k=1$ and any $\text{dim}\T$,
(iii) $k=2$ and $\text{dim}\T=3$. This does not affect the rest of
the book, except that reader must use (practically always) the
$C_+$-sense for positiveness.}


\


\section{Proofs}


I will not give here a proof of the first part of the local theorem
by two reasons: first it is real pain to write and read and second it is not really mine.
The proof I have is just a modification of one unpublished Perelman result.
Since this result of Perelman was never published and (as far as I know) was never written,
I put it in the Appendix B (It is more fun to look at the original proof then at my compilations).


\parit{Proof of the second part of Local Theorem.}
Let us prove first that
if $(M,g)$ is a Riemannian manifold with $\cosec M>0$
then $(M,g)$ is isometric to a convex submanifold in $\RR^q$.
This is equivalent to the fact that there is an isometric embedding
$i\:M\to \RR^q$, such that
$\Phi(i)>0$ (see 1.D).

In general, a smooth isometric embedding of $(M,g)$ with $\cosec M>0$
may have undefined $\Phi$-tensor, but there is a way to make it positive.

Consider any smooth free isometric embedding  $i\:(M,g)\to\RR^q$, then by
Theorem [Grom]${}_{\text{PDR}}$ 3.1.5(A) for any tensor field $E\i \S^4(\T)$
such that $E_x>0$ (see 1.E) at all $x\in M$
one can find a $C^1$-close isometric embedding
$i'\:(M,g)\to \RR^q$, such that $E(i')=E(i)+E$.
In particular, one may choose $E=cg^{\circ 2}$ for any $c>0$.
Since $\cosec M>0$ the number $c$ can be chosen  big enuf so that
at $\Phi(i')>0$.

(Let us describe a more direct way to construct $i'$ with $E(i')=E(i)+E$ for $E=c\cdot g\circ g$.
First construct an isometric embedding $j\:\RR^q\to \RR^{q'}$,
such that
$E(j)=c\cdot h^{\circ 2}$, where $h=\sum_{i=1}^q (dx_i)^2$
is the unit $2$-form on $\RR^q$ and then take $i'=j\circ i$.
One can construct $j$ on the following way:
first choose a collection of linear functions $l_i\:\RR^q\to \RR$
such that $h^{\circ 2} = \sum_i (dl_i)^4$ then take diagonal
of product of the following mappings: a linear mapping $L\:\RR^q\to \RR^q$
and twists $\tau_i\: \RR^q\to \RR^2$,
\[\tau_i(x)=
\bigl(a_i\cdot  \sin \bigl(b_i\cdot l_i(x)\bigr),a_i\cdot  \cos \bigl(b_i\cdot l_i(x)\bigr)\bigr)\]
with appropriately chosen $a_i$ and $b_i$.)

Now, since $M\i \RR^q$ is a convex submanifold there is an open
set $U\i M$ which is an intersection of open convex hyper-surfaces
$F_i$ with obtuse angle between each pair of outward normals
everywhere on $U$.
Approximate each $F_i$ as a convex polyhedral hyper-surface $F_i^\epsilon$ keeping the angles obtuse.
The intersection $U^\epsilon$ of all $F_i^\epsilon$ is obviously polyhedral submanifold
and it has curvature $\ge 0$ (see B). Cutting subdomains from
$U^\epsilon$ if necessary one gets the needed approximation. \qeds


\parit{Proof of Global Theorem.} Let us first assume that
$\T M$ is stably trivial.
We will represent our submanifold $M$ as
an intersection of (open) convex hyper-surfaces with angles between any pair of outward normals $>\pi/2$
everywhere on $M$.
Tio finish the proof, we repeat the construction from  Local Theorem.

Existence of such representation is obviously equivalent to the
existence of a smooth section of orthonormal bases $\{e_i\}$ in
$\N\, M$ such that $s=\sum_i s_i e_i$ with positively defined
$s_i\in \S^2(\T)$.

Consider a cover $U_k$, $k\in \{1,2,..,n\}$
of $M$ such that on each $U_k$ there is a smooth section of
orthonormal bases $\{e_{i,k}\}\i \N\,M$ with the above properties.
Since $\T M$ is stably trivial we can assume that $\N\,M$ is a trivial bundle.
Therefore we
can extend these bases to all $M$, and get $n$ bases
$\{e_{i,k}\}$ for all $\N\,M$.
Therefore at each point we have an isometric rotation $E_{k,k'}\in O(q-m)$
which sends $\{e_{i,k}\}$ to $\{e_{i,k'}\}$.
Without loss of generality we can assume that correspondent mapping
$E_{k,k'}\:M\to O(q-m)$ is homotopic to a trivial one.
Now let us take a smooth partition of unity $u_k\:M\to [0,1]$,
$u_k|_{M\backslash U_k}\equiv 0$ and
$\sum_k u_k(x)\equiv 1$ for all $x\in M$.
At each point $x\in U_k\i M$ we have
$\Phi_x\equiv \sum_i s_{i,k}^2$.
Therefore for each $x\in M$ we have
$\Phi_x\equiv \sum_{i,k} u_k(x) s_{i,k}^2$.
Consider $n\N\,M=\N_1 M\oplus \N_2 M\oplus ...\oplus \N_n M$,
the sum of $n$ copies of the normal bundle,
take the basis $\{e_{i,k}\}$ for $\N_k$, and consider the subbundle $\N_\Delta(M)$,
which is spanned by
$(\sqrt u_1 e_{i,1}, \sqrt u_2 E_{12}e_{i,1},..., \sqrt u_n E_{1n}e_{i,1})$.
It is obviously a trivial subbundle with trivial orthogonal subbundle.
Therefore, if one considers $\RR^{(n-1)(q-m)}\times \N\,M$ then there is
a bundle isomorphism $i\:n\N\,M\to \RR^{(n-1)(q-m)}\times \N\,M$
which is an isometry on each fiber, which sends
$\N_\Delta M$ to $\N\,M$ and moreover
if $p_\Delta\:\N^n\to \N_\Delta$ is the orthogonal projection then $i\circ p_\Delta (e_{i,k})=\sqrt u_k e_{i,k}$.

Therefore, we get a smooth section of orthonormal bases
$\{e_{i,k}\}\i \N'M=\RR^{(n-1)dim(\N_x)}\times \N\, M$;
that is if we had $\N\,M$ as a normal bundle of $M\i \RR^{q}$ then $\N'M$ is a
normal bundle of $M\i \RR^{q}\times 0\i \RR^{q}\times
\RR^{(n-1)(q-m)}$. Now for each pair of indexes $i,k$ we have a
nonnegative quadratic form $s_{i,k}= \langle s,e_{i,k}\rangle$,
$\Phi_x\equiv \sum_{i,k} u_k(x) s_{i,k}^2$ and at each point we
have at least one quadratic form which is strictly positive. It is
not hard to rotate basis $e_{i,k}$ a little to get a new smooth
section of bases in $\N'M$ with representation $s= \sum_{i,k}
e_{i,k}(s'_{i,k})$ where each $s'_{i,k}$ is strictly positive.

If $\T M$ is not stably trivial one still can find an embedding $M\to \RR^q$
which has positive $\Phi$-curvature at each point.
Take a small tubular neighborhood $U$ of $M$.
Let $\tilde M$ be a finite cover of $M$ such that $\T \tilde M$ is stably trivial.
 We can assume that $\N\,\tilde M$
is trivial, therefore $\N\,M$ is equivalent to a flat bundle.
From the above we get existence of
flat bundle $U'\to U$ such that the new induced normal bundle of $M$ with
respect to $U'$ is trivial. $U'$ is an open flat manifold and that
makes possible to repeat the same construction as above.
\qeds



\section{Remarks on curvature bounded above}




One can also ask a similar question for approximations by polyhedral spaces 
with upper curvature bound (in the sense of Alexandrov). The similar answer can be obtained on the similar way.


\begin{thm}{Local Theorem} Let $P_n$ be a sequence of $m$-dimensional
polyhedral spaces
with curvature $\le \kappa$, which Lipschitz converge to a
Riemannian manifold $(M,g)$ of the same dimension, then $\cosec M\le \kappa$.
\end{thm}

Unfortunately, I can not find as nice characterization for the Global Theorem
as in the case of curvature bounded below. Here is what I can do:


\begin{thm}{Global Theorem.}
 If $(M,g)$ is Riemannian $m$-manifold with $\cosec M\le \kappa$.
Assume that $M$ 
is diffeomorphic to a direct product of manifolds which care constant 
negative curvature, then $M$ can be realized as a Lipschitz limit
of sequence of $m$-dimensional polyhedral metrics with curvature
$\le \kappa+\epsilon$ for arbitrary $\epsilon>0$.
\end{thm}

The proof of local theorem is practically the same as for curvature bounded 
below. Perelman's Lemma (which is the main technical tool in the proof) is 
also true for negative curvature, 
(infact it it true in all lenght-metric spaces which have 
enuf convex functions).

The proof of the global theorem is also very similar, but one should consider 
embeddings into (non complete) flat pseude Riemannian manifold, which is locally isometric to $\RR^{m,q}$  as a space-like submanifolds of maximal dimension $m$ with trivial normal bundle.
 Unfortunately it is not clear which manifolds admit such an embedding, but it is clear that if it does then the universal cover of $M$ is diffeomorphic to $\RR^m$. 
In particular $M$ is $K(\pi,1)$ space, but so far I can not say much useful 
about $\pi$. On the other hand if $M$ cares a metric of constant negative curvature then it is a factor of pseudosphere in $\RR^{m,1}$ along some group,
and factor of a little neighborhood of the pseudosphere along this subgroup gives the needed ambiant manifold. Taking product of such manifolds one has the needed embeddings for products of such manifolds.













\section{Problem section}


The opposite question;
that is, which polyhedral metrics
could be smoothed to a Riemannian manifold with positive curvature,
is still open. All examples I know so far meet the following


\begin{thm}{Conjecture} Any polyhedral metric with curvature $\ge \kappa$
can be smoothed in to a Riemannian orbifold with cosectional
curvature $\ge \kappa-\epsilon$.
\end{thm}

In case of dimension $=2$ the above Conjecture is a trivial
corollary of Alexandrov's embedding theorem [Al]. In dimension
$=3$, one can smooth each vertex using the same Alexandrov's
embedding theorem (this time we need embedding of surface of
curvature $\ge 1$ into $\SS^3$) and one can do the smoothing in such
a way that the only singular points left will be ``midpoints of
edges'' and these singular points are conic. Then one can smooth
the remaining points in the same way.

The conjecture would imply, in particular,
that any simply connected
manifold with positive cosectional curvature is diffeomorphic
to a sphere.
Indeed the Corollary 0.3
implies that if $M$
is a Riemannian manifold with $\cosec M\ge 1$ then it can be approximated by polyhedral spaces $X_n$
with curvature $\ge 1$.
 Therefore the spherical suspension $\SS(M)$ is approximated by $\Sigma (X_n)$ (cf. [GW]).
Now from the conjecture it would follow that $\Sigma(X_n)$ is smoothable into Riemannian orbifold and therefore $M$ is a quotient of a sphere.

\

An other question is whether the condition of stably trivial tangent bundle
can be removed from Theorem 0.2.
 So far, I can not even construct an approximation of
$(\CP^2,\text{can})$ by polyhedral metrics with curvature
$\ge -\epsilon$.

One may ask whether it is possible to construct an approximation
of $(\CP^2,\text{can})$ by polyhedral metrics with curvature
$\ge 0$. This is already a rigid question, in particular, from
Cheeger's results [Ch] it is easy to see that any nonnegatively
curved polyhedral metric on $\CP^2$ carries complex structure.
As it was pointed out by Mikhail Gromov,
$\CP^2$ carries polyhedral
metrics with curvature $\ge 0$ 
[for example, take nonnegatively
curved polyhedral metric on $\SS^2$ then the space of all pairs of
points in $\SS^2$ homeomorphic to $\CP^2$ and naturally comes with
nonnegatively curved polyhedral metric] altho it is not clear
whether such metrics can approximate canonical metrics on $\CP^2$, (see [Pan] for more examples and general discussion of 
polyhedral spaces with complex structure). 


Is there any way to generalize Alexandrov embedding theorem? For
example, is it possible to characterize Riemannian manifolds which
are isometric to a complete convex hyper-surface in a complete
convex hyper-surface in ... in $\RR^q$? 
Is it true that any simply
connected Riemannian manifold $M$ with $\cosec M>0$ is isometric to one of
those? 
Again, if this manifold is compact it would immediately
give that any such manifold is diffeomorphic to standard sphere.


Is it possible to charactarise $m$-manifolds which adimits an embedding 
into flat open $(m,q)$-pseudo-Riemannian maniflod as a space-like surface?
(It is easy to see that if such an embedding exist then 
the univeral cover of $M$ is diffeomorphic to $\RR^m$, plus the first homotopy group must be linear, but I do not think it is enuf for existance of such embedding.)













\appendix
\section{Example of positive curvature tensor which cosectional curvature is not positive}




Here I present calculations of J.Eschenburg, which show that
curvature tensor $R$ of $\SU(3)$ with bi-invariant metric has
non-negative curvature operator but it is not true that $\cosec M\ge 0$.
This gives an example for dimension $\ge 8$, from the work of
Zoltek [Zol] it follows that such examples exist for dimension
$\ge 5$ but the calculations below much simpler and I hope that it
will be useful for a reader who wants quickly convince him-self
that such monsters do live.

\

Consider Lie algebra $su(3)$, and adjoint representation
ad$\:su(3)\to \Lambda^2(su(3))$. The curvature operator of $\SU(3)$
with bi-invariant metric has curvature operator
$R\:\Lambda^2(su(3))\to \Lambda^2(su(3))$ which coincide with
projection on Im$(ad)$.

Now if one can prove that $Im(ad)$ has no simple bi-vector inside
then it will follow that curvature operator of $\SU(3)$ with bi-invariant metric
does not have positive cosectional curvature.

Therefore we only have  to show that if $0\not=x\in su(3)$ then
$ad_x\in \Lambda^2(su(3))$ is not a simple bi-vector, i.e $\not=
v\wedge w$.

It is sufficient to prove it for $ad_x$, where $x$ is tangent to a
maximal torus of diagonal elements in a matrix representation.
Therefore in the matrix representation it looks like
$x=diag\{ai,bi,ci\}$ with $a+b+c=0$. Take the standard real basis
in $su(3)$, which comes from matrix form;
that is, take
 $A_1=diag \{i,0,-i\},A_2=diag \{0,i,-i\}$, take
 $F_1=e_2\wedge e_3,F_2= e_3\wedge e_1,F_3=e_1\wedge e_2$ be real and
$E_1=i e_2\circ e_3,E_2=i e_3\circ e_1,E_3=i e_1\circ e_2$
imaginary parts of basis, here $e_1,e_2,e_3$ is a basis of $\CC^3$
where $\SU(3)$ acts.

Then by the direct calculation we have
ad$_x=(c-b)F_1\wedge E_1+(a-c)F_2\wedge E_2 + (b-a) F_3\wedge E_3$,
now the fact that bi-vector $\phi\in \Lambda^2(\T)$
is simple is equivalent to $\phi\wedge\phi=0$, and
$$ad_x\wedge ad_x=$$
$$=(c-b)(a-c)F_1\wedge E_1\wedge F_2\wedge E_2
+
(a-c)(b-a)F_2\wedge E_2\wedge F_3\wedge E_3
+$$
$$+(b-a)(c-b)F_3\wedge E_3\wedge F_1\wedge E_1$$
Therefore if ad$_x$ is simple then at least two
of numbers $(c-b),(a-c),(b-a)$ are zeros
and since $a+b+c=0$ we have that  $a=b=c=0$; that is, $x=0$. \qeds



\section{A theorem of Perelman and why I need it.}


In this appendix I will present the proof of one unpublished
result of G.Perelman, on which idea I build the proof of the first
part of Local Theorem.

Let $M$ be an Alexandrov $m$-space and $U\i M$ be an open subset.
Let $F\:U\to \RR^m$ be a chart $F(p)=(x_1(p),x_2(p),...,x_m(p))$.
We say that $F$ is convex if each of co-ordinate functions $x_i$
is convex. The proof of the following claim easily follows from
Proposition 3 [Per]


\begin{thm}{B.1 Claim} Let $g$ be a convex function on $U$ and for some
convex chart $F\:U\to \RR^m$ we have $\partial g/\partial x_i< 0$
then $g\circ F^{-1}$ is a convex function on $F(U)$. Moreover for
any $p\in U$ and  $v\in \T_p$ we have
$$\nabla^2_vg\le\nabla^2_{dF(v)}\bigl(g\circ F^{-1}\bigr).$$
\end{thm}

In particular, if $S$ is a level surface of $g$ and $F$ be a short
chart then for any $p\in S$
$$I\!I_S(X,X)\le I\!I_{F(S)}(dF(X),dF(X)).$$


\begin{thm}{B.2 Definition.} Let $X_n \buildrel {GH} \over \rightarrow X$ be a
converging sequence of metric spaces and $f_n\:X_n\to X$ be correspondent
sequence of Hausdorff approximations.
We say that a sequence of measures $\mu_n$ on $X_n$ weakly converge to
measure $\mu$ on $X$ if for any continuous function $\alpha$ with compact
support on $X$ we have $\int_{X_n} \alpha\circ f_nd\mu_n\to \int_{X} \alpha d\mu$.
\end{thm}


The proof of the following result I heard from G. Perelman about
seven years ago. The proof below should be close to the original
but some ideas might differ.


\begin{thm}{B.3 Theorem.} Let $M_n$ be a sequence of Riemannian $m$-manifolds with curvature
$\ge \kappa$ which Lipschitz converge to a closed Riemannian manifold $M$.
Then scalar curvature on $M_n$ converges weakly to the scalar curvature on $M$.
(that is, $Sc_{g_n}dv_{g_n}$ converges weakly to $Sc_{g}dv_{g}$).
\end{thm}


Let us note that if one has no lower bound for curvature then
there are examples when limit of Scalar curvatures is smaller than
Scalar curvature of the limit, and it is unknown whether it could
also be bigger.


Let us prepare the following Lemma (which is in fact a partial case of the theorem):


\begin{thm}{B.4 Lemma.}
 Let $F_n$ be a sequence of smooth convex hyper-surfaces in $\RR^{m+1}$ which Hausdorff converges to a smooth convex hyper-surface $F$.
 Let $Sc(F)$ and $Sc(F_n)$ denote scalar curvatures of $F$ and $F_n$
and $h(F)$, $h(F_n)$ denote $m$-Hausdorff measure of the
correspondent hyper-surface. Then $Sc(F_n)dh(F_n)$ converges weakly to $Sc(F)dh(F)$
\end{thm}


\parit{Proof of Lemma B.4.} Let $\alpha$ be a continuous function
with compact support in $\RR^{m+1}$. Let us denote by $C_r(F)$ the
set of points in $\RR^{m+1}$ which lie on outgoing normal rays to
the hyper-surface $F$ on the distance $< r$ to the hyper-surface.
Let us define $\alpha_F\:C_\infty(F)\to \RR$,
$\alpha_F(x)=\alpha(y)$ where $y\in F$ is a closest point on the
hyper-surface.

Now, it is well known and easy to see that
$\int_{C_r(F)}\alpha_Fdv$ is a polynomial of degree $m$ on $r$,
moreover, the quadratic term is exactly $r^2\int_F\alpha Sc(F)
dh(F)$

Now if we have $F_n\to F$, then $C_r(F_n)$ converge to $C_r(F)$
and $\alpha_{F_n}$ converge to $\alpha_{F}$. Therefore,
$\int_{C_r(F_n)}\alpha_{F_n}dv\to\int_{C_r(F)}\alpha_Fdv$ and the
 coefficient with $r^2$ of correspondent polynomials also converges.
That finishes the proof.\qeds


\parit{Proof of Theorem B.3.} We first want to construct special
distance-like charts in a neighborhood of any point in $M$
together with some nice approximating charts on $M_n$.


\begin{thm}{B.5 Lemma} Given $p\in M$, $v\in T_p^*(M)$ and $\epsilon>0$
there is $\delta>0$ and sequence $M_n\ni p_n\to p\in M$ and
sequence of convex functions $f_n\:B_\delta(p_n)\subset M_n\to \RR$
which converges to a convex function $f\:B_\delta(p)\subset M\to
\RR$ such that $d_pf=v$, $|f''|<\epsilon$ everywhere on
$B_\delta(p)$.
\end{thm}


\parit{Proof of Lemma B.5.} Consider an orthonormal basis $\{e_i\}$
in $\T_p M$ such that $\sum_i e_i=cv$. Take $r >0$ an let
$a_i=\exp_p r e_i$. Now $f=\sum_i \phi\circ\text{dist}_{a_i}$,
where $\phi(x)=\alpha \log x-\beta x^2$ if dim$(M)=2$ and
$\phi(x)=\alpha \frac{1}{x^{n-2}}-\beta x^2$ if dim$(M)>2$. The
same arguments as in [PP]4.3 show that $f$ satisfy the conditions
in the Lemma for appropriately chosen $\alpha$ and $\beta$, in a
small ball $B_\delta(p)$.

Now to construct an approximation of this function construct a
sequence $a_{i,n}\to a_i$ for each $i$ and take $f_n=\sum_i
\phi\circ\text{dist}_{a_{i,n}}$. Again the same reasoning as in
[PP]4.3 proves that there is $\epsilon>0$ such that for large $n$
the function is convex in a $\delta$-neighborhood of $p_n$.\qeds


Now we may take any orthonormal basis ${v_i}\subset \T^*_p M$ and
construct a function $f_i\:B_\delta(p)$ together with an approximations
$f_{i,n}\:B_\delta(p_n)$. In addition to the above properties these
functions will be almost orthogonal for a small enuf $\delta$;
that is, one can assume
that angle between level surfaces lies in between $\pi/2\pm\epsilon$.


Now we start induction by dimension, we can take dim$=2$ as a base, in which case convergence follows from Gauss--Bonnet formula. Now assume we already proved it for all dimensions $<m$.

To save space/time on the notation let us agree that extra index
$n$ will always denote correspondent babe for $M_n$.

Let $p\in M$ and $S_1,S_2,...,S_m$ be one-parameter families of
 co-ordinate surfaces $f_i=c$.
Let us denote by $Sc_i$ is the ``scalar'' curvature of directions tangent  to $S_i$, in other words $Sc_i=Sc-Ricc(u_i)$ where $u_i$ is unit vector field normal to $S_i$.
Note that from lower curvature bound we have $|\text{Rm}|<c_1+c_2Sc$
and therefore $(1+\alpha)(m-2)Sc=Sc_1+Sc_2+...+Sc_m$
where $\alpha$ depend on angles between these co-ordinate
surfaces and $\alpha\to 0$
as all these angles converge to $\pi/2$, in particular as $\epsilon\to 0$.


Let $Sc(S_i)$ be the scalar curvature of the intrinsic metric of the correspondent
co-ordinate surface. Since the Jacobian of our charts converges to the Jacobian of
the limit chart from the induction hypothesis we have $Sc({S_{i,n}})dv_{g_n}$
converges weakly to $Sc(S_i)dv_g$.

From Gauss formula we have $Sc_i+G(S_i)= Sc(S_i)$, where
$G(S_i)=\sum_{i\not=j}k_ik_j$ where $k_i$ are the principal
curvatures of $S_i$. Since each $S_i$ is convex
$$Sc_i\le Sc(S_i)\ge G(S_i).$$
Therefore, after passing to a subsequence
$Sc_{i,n}$ should converge weakly to some
$\bar{Sc}_i\le Sc(S_i)\le Sc_i+n(n-1)\epsilon^2$.

Now let us prove the following lower bound: $\bar {Sc}_i\ge  Sc_i-C\epsilon^2$ for some $C=C(m)$.

The inequality above also insure that after passing to a subsequence $G(S_{i,n})$
 converges to some measure $\bar G(S_i)$ and obviously
$\bar Sc_i=Sc(S_i)-\bar G(S_i)\ge Sc_i-\bar G(S_i)$
therefore it is enuf to show that $\bar G(S_i)\le C\epsilon^2$ for some fixed $C$.

To give this last estimate let us construct a new chart similar to
one before, $H=(h_1,h_2,...,h_m)$ with the approximations
$H_n=(h_{1,n},h_{2,n},...,h_{m,n})$ such that $\partial f/\partial
h_i<-1/10m$. From the Claim above we have that $G(S_n)\le c
G(H(S_n))$ as well as $G(S)\le c G(H(S))\le C\epsilon^2$

Now $H(S_n)$ converges to $H(S)$ as convex hyper-surfaces in $\RR^m$
and applying Lemma we get $G(H(S_n))$ converges weakly to $G(H(S))$.

Since for any $\epsilon>0$ there is a finite covering of $M$ by charts as in Lemma B.5 we obtain the Theorem. \qeds


Along the same lines one can prove stronger statements:

\begin{thm}{Smooth Proposition} Let $(M_n,g_n)$ be a sequence of
Riemannian $m$-manifolds with curvature $\ge \kappa$ which
GH-converges to a Riemannian manifold $(M,g)$ of the same
dimension $=m$. Then there is a sequence reparameterizations
(diffeomorphisms) $f_n\:M\to M_n$, such that curvature tensor of
$df_n^*(g_n)$ weakly converges to the curvature tensor of $g$ on
$M$.
\end{thm}


\begin{thm}{Corollary} Let $R$ be an $\SO(\T)$ invariant convex set in $\A^4(\T)$. Assume
that there is a lower bound $\kappa>-\infty$ for sectional curvature in $R$.
Let $M_n$ be a sequence of Riemannian manifolds with curvature tensor from $R$ at each point which converges to a Riemannian manifold $M$ of the same dimension. Then the curvature tensor at any point of $M$ is from $R$.
\end{thm}


For example, smooth limit of manifolds with positive curvature
operator must have positive curvature operator.

 Note that this corollary can not hold for general
$\SO(\T)$ invariant convex set in $\A^4(\T)$, for example as it shown
in [Loh1], [Loh2] it is not true for sets $R=\{r\in
\A^4\:Ricci(r)\le c\}$ curvature and for $R'=\{r\in \A^4\mid c\le
Sc(r)\le c+\epsilon\}$.
Altho I believe it should be still true
with much more relaxed limitations on $R$.




Finally one can give a singular version of this result which we need in our paper:

First let us describe the singular curvature tensor of a polyhedral space. 
Assume
we have a $1\pm\epsilon$-bi-Lipschitz parametrization of polyhedral $P$
by smooth
Riemannian manifold $f\:M\to P$, such that $f^{-1}$ is smooth on each simplex.
One can think about $P$ as $(M,d)$ where $d$ is a singular metric.
Now let us define the curvature tensor of $d$ as following: it's support is the
 image of $(n-2)$-skeleton of $P$ and on each $(n-2)$ simplex it is defined as
$h_{n-2}(2\pi-\omega)\alpha$ where $h_{n-2}$ is the Hausdorff
measure of this image, $\omega$ is the total angle around this
simplex $\Delta$ and $\alpha=dx\wedge dy$ is a bi-vector field
with the following properties: $|\alpha|=1$ everywhere on the
image of simplex and $\alpha|_{f^{-1}(\Delta)}=0$.





\begin{thm}{Singular Proposition.} Let $P_n$ be a polyhedral $m$-spaces
with curvature $\ge \kappa$ which GH-converges to a Riemannian
manifold $(M,g)$ of the same dimension $=m$. Then there is a
sequence smooth parameterizations $f_n\:M\to P_n$, such that the
described singular curvature tensor  weakly converges to the
curvature tensor of $g$ on $M$.
\end{thm}


As well as in the corollary above, since the cosectional curvature of $(M,d_n)$ is positive, we get that the curvature tensor on the limit $(M,g)$ has positive cosectional curvature
and that proves the first part of the Local Theorem 0.1.



\section*{References}


[Al] Aleksandrov, A. D. Vypuklye mnogogranniki.
(Russian) [Convex Polyhedra] Gosudarstv. Izdat. Tehn.-Teor. Lit.,
Moscow-Leningrad, 1950. 428 pp.

[Ch] Cheeger, Jeff
On the spectral geometry of spaces with cone-like singularities.
Proc. Nat. Acad. Sci. U.S.A. 76 (1979), no. 5,
2103--2106.

[Grom]${}_{\text{PDR}}$ Gromov, Mikhael Partial differential relations.
Ergebnisse der Mathematik und ihrer Grenzgebiete
(3) [Results in Mathematics and Related Areas (3)],
9. Springer-Verlag, Berlin-New York, 1986. x+363 pp.
/Russian translation ``Mir'', Moscow, 1990. 536 pp.

[Grom]${}_{\text{SGMC}}$ Gromov, M.
Sign and geometric meaning of curvature. (English. English, Italian summary)
Rend. Sem. Mat. Fis. Milano 61 (1991), 9--123 (1994).

[GW] Grove, Karsten; Wilhelm, Frederick Metric constraints on exotic spheres via Alexandrov geometry. J. Reine Angew.
Math. 487 (1997), 201--217

[Hil] Hilbert, D.
\"Uber die Darstellung definiter Formen als Summe von Formenquadraten.
Mathem. Annalen Bd 32, S. 342-350 (1888), also in Hilbert, D.
Gesammelte Abhandlungen. Zweiter Band. Algebra,
Invariantentheorie, Geometrie. (German) Chelsea Publishing Co.,
New York 1965 viii+453 pp.

[Loh]$_{GLC}$ Lohkamp, Joachim Global and local curvatures.
Riemannian geometry (Waterloo, ON, 1993), 23--51, Fields Inst.
Monogr., 4, Amer. Math. Soc., Providence, RI, 1996.

[Loh]$_{SCH}$ Lohkamp, Joachim Scalar curvature and hammocks.
Math. Ann. 313 (1999), no. 3, 385--407.


[MM]
M. J. Micallef; J.D. Moore
Minimal two-spheres and the topology of manifolds
with positive curvature on totally isotropic two-planes
Ann. of Math.
127 pp.199-227
1988


[Mil] Milka, A. D.
Multidimensional spaces with polyhedral metric of nonnegative curvature. I.
(Russian) Ukrain. Geometr. Sb. Vyp.
and
Multidimensional spaces with polyhedral metric of nonnegative curvature. II.
(Russian) Ukrain. Geometr. Sb. No. 7,
(1969), 68--77, 185 (1970).

[Pan] Panov, D. Polyhedral Kähler manifolds. Geom. Topol. 13 (2009), no. 4, 2205--2252.

[Per] G.Perelman, DC Structure on Alexandrov Space, preprint

[PP] Perelman, G. Ya.; Petrunin, A. M.
Extremal subsets in Aleksandrov spaces and the generalized Liberman theorem.
(Russian) Algebra i Analiz 5 (1993), no. 1,242--256;
translation in St. Petersburg Math. J. 5 (1994), no. 1, 215--227

[Pet] Petrunin, Anton Applications of quasigeodesics and gradient
curves. Comparison geometry (Berkeley, CA, 1993--94), 203--219,
Math. Sci. Res. Inst. Publ., 30

[Zol] Zoltek, Stanley M. Nonnegative curvature operators: some nontrivial examples. J. Differential Geom. 14 (1979), no. 2,
   303--315.

\end{document}
